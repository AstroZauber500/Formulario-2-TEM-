\documentclass[11pt]{article}
\usepackage[utf8]{inputenc}
\usepackage{amsmath, amssymb}
\usepackage{geometry}
\geometry{margin=2cm}
\usepackage{graphicx}
\usepackage{hyperref}
\usepackage{enumitem}

\title{Formulario de Magnetostática}
\author{Tu Nombre}
\date{\today}

\begin{document}

% =========================
\section{Ley de Biot–Savart (en términos de \(\mathbf{H}\))}
% =========================
\subsection*{Forma diferencial}
\[
d\mathbf{H}
  =\frac{I\,d\mathbf{l}\times\hat{\mathbf{R}}}{4\pi R^{2}}
\qquad
(\hat{\mathbf{R}}=\mathbf{R}/R)
\]

En coordenadas cilíndricas \(({\rho},\phi,z)\) para un filamento sobre el eje \(z\),
\[
d\mathbf{H}= \frac{I\,d\mathbf{l}l}{4\pi R^{2}}
             \,\hat{\boldsymbol{a}}_{{\rho}}
             \times\hat{\mathbf{R}}
\]

% =========================
\subsection*{Contribuciones de corrientes}
% =========================
\begin{align}
\mathbf{H}(\mathbf{r})
  &=\frac{I}{4\pi}\int_{\text{línea}}
      \frac{d\mathbf{l}\times\hat{\mathbf{R}}}{R^{2}}
      &&\text{(corriente lineal \(I\))}\\[6pt]
  &=\frac{1}{4\pi}\int_{\text{superficie}}
      \frac{\mathbf{K}\times\hat{\mathbf{R}}}{R^{2}}\,
      dS
      &&\text{(corriente superficial \(\mathbf{K}\))}\\[6pt]
  &=\frac{1}{4\pi}\int_{\text{volumen}}
      \frac{\mathbf{J}\times\hat{\mathbf{R}}}{R^{2}}\,
      dV
      &&\text{(corriente volumétrica \(\mathbf{J}\))}
\end{align}

\subsection*{Fuerza de Lorentz}
\begin{align}
\mathbf{F}=Q\,\mathbf{u}\times\mathbf{B}
\end{align}

% =========================
\section{Campo \(\mathbf{H}\) de un filamento finito en el eje \(z\)}
% =========================
Sea un conductor recto de longitud \(2L\) centrado en el origen,
transporte de corriente \(I\) hacia \(+z\).
Para un punto \(P({\rho},0,0)\) en el plano \(xy\):

\[
\mathbf{H}(r)=
\frac{I}{4\pi {\rho}}\left(
  \sin\alpha_{2}-\sin\alpha_{1}
\right)\,\hat{\boldsymbol{a}}_{\phi},
\]
donde
\(\alpha_{1}\) y \(\alpha_{2}\) son los ángulos
entre el segmento \(P\)‑extremo y el eje \(r\).

\textbf{Casos límite}
\begin{itemize}
  \item {\bf Semi‑infinito} \((L\to\infty\) solo en \(+z)\):
        \(\alpha_{1}=0,\;\alpha_{2}=\pi/2\)
        \[
        \boxed{\mathbf{H}(r)=\dfrac{I}{4\pi {\rho}}\,\hat{\boldsymbol{a}}_{\phi}}
        \]
  \item {\bf Infinito} \((L\to\infty\) en ambas direcciones):
        \(\alpha_{1}=-\pi/2,\;\alpha_{2}=+\pi/2\)
        \[
        \boxed{\mathbf{H}(r)=\dfrac{I}{2\pi {\rho}}\,\hat{\boldsymbol{a}}_{\phi}}
        \]
\end{itemize}

% =========================
\section{Ley de Ampère}
% =========================
\subsection*{Enunciado}
La ley de Ampère establece que la integral de línea de \(\mathbf{H}\)
alrededor de un camino cerrado equivale a la corriente neta \(I_{\mathrm{enc}}\)
encerrada por dicho camino:
\[
\oint_{\mathcal{C}}\mathbf{H}\cdot d\boldsymbol{\ell}= I_{\mathrm{enc}}.
\]

\subsection*{Forma inversa}
\[
I_{\mathrm{enc}}=\oint_{\mathcal{C}}\mathbf{H}\cdot d\boldsymbol{\ell}.
\]

% =========================
% =========================
\subsection*{Aplicaciones típicas (con unidades)}
% =========================
\begin{itemize}
  \item \textbf{Hilo recto infinito}  
        \[
        \mathbf{H} = \frac{I}{2\pi {\rho}}\,\hat{\boldsymbol{a}}_{\phi} \quad [\text{A/m}]
        \]
        Usando contorno circular alrededor del hilo.

  \item \textbf{Solenoide largo}  
        \[
        \mathbf{H} = nI\,\hat{\boldsymbol{a}}_{z} \quad [\text{A/m}]
        \]
        Donde \(n\) es el número de vueltas por unidad de longitud.

  \item \textbf{Toroide}  
        \[
        \mathbf{H} = \frac{NI}{2\pi {\rho}}\,\hat{\boldsymbol{a}}_{\phi} \quad [\text{A/m}]
        \]
        Siendo \(N\) el número total de espiras.

  \item \textbf{Lámina infinita de corriente (densidad superficial \(K\))}  
        \[
        \mathbf{H} = \frac{K}{2}\,\hat{\boldsymbol{a}}_{\parallel} \quad [\text{A/m}]
        \]
        Donde \(K\) es la densidad de corriente superficial en [A/m] y el campo es paralelo a la lámina, orientado según la regla de la mano derecha.

  \item \textbf{Línea de transmisión coaxial infinita}  
        \[
        \mathbf{H}({\rho}) = 
        \begin{cases}
        \dfrac{I\,{\rho}}{2\pi a^{2}}\,\hat{\boldsymbol{a}}_{\phi} & \text{si } {\rho} < a \\
        \dfrac{I}{2\pi {\rho}}\,\hat{\boldsymbol{a}}_{\phi} & \text{si } a < {\rho} < b \\
        0 & \text{si } {\rho} > b
        \end{cases}
        \quad [\text{A/m}]
        \]
        Donde \(a\) y \(b\) son los radios interno y externo, respectivamente, de los conductores.
\end{itemize}

% =========================
\section{Flujo de Campo Magnético}
% =========================
\subsection*{Definición}
El flujo magnético total que atraviesa una superficie \(S\) es:
\[
\Phi_{B} = \int_{S} \mathbf{B} \cdot d\mathbf{S}
\quad [\text{Wb}]
\]
Donde:
\begin{itemize}
  \item \boxed{\mathbf{B} = \mu_0 \mathbf{H})} es la densidad de flujo magnético \([\text{T} = \text{Wb/m}^2]\).
  \item \(d\mathbf{S}\) es el vector diferencial de área orientado perpendicular a la superficie.
\end{itemize}

\subsection*{Propiedad fundamental}
\[
\nabla \cdot \mathbf{B} = 0 \quad \Rightarrow \quad
\oint_{S} \mathbf{B} \cdot d\mathbf{S} = 0
\]
El flujo total a través de una superficie cerrada es cero. No existen monopolos magnéticos.


% =========================
\section{Potencial Magnetico \(\mathbf{A}\) y $V_m$ }
% =========================
\subsection{Potencial magnético escalar $V_{m}$:}

\[\mathbf{H}= -\nabla V_m,\qquad si \qquad\mathbf{J} = 0\]

\subsection{Potencial magnético vector \(\mathbf{A}\):}

\[\]
\[\mathbf{A}= \int_{L}\frac{\mu Id\mathbf{l}}{4\pi R}
\]

\[
\mathbf{B}=\mu_0\mathbf{H}=\nabla\times\mathbf{A},
\qquad
\mathbf{A}(\mathbf{r})=\frac{\mu_0}{4\pi}\int
\frac{\mathbf{J}\,dV}{R}.
\]
Para un hilo infinito:
\[
\mathbf{A}= \frac{\mu_0 I}{2\pi}\ln\!\left(\frac{r}{r_0}\right)\,
\hat{\boldsymbol{a}}_{\phi}.
\]

%\end{document}





\end{document}
