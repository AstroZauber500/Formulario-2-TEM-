\documentclass[11pt]{article}
\usepackage[utf8]{inputenc}
\usepackage{amsmath, amssymb}
\usepackage{geometry}
\geometry{margin=2cm}
\usepackage{graphicx}
\usepackage{hyperref}
\usepackage{enumitem}
\usepackage{float}
\usepackage{multicol}


\title{Formulario de Magnetostática}
\author{Tu Nombre}
\date{\today}

\begin{document}

% =========================
\section{Ley de Biot–Savart (en términos de \(\mathbf{H}\))}
% =========================
\subsection*{Forma diferencial}
\[
d\mathbf{H}
  =\frac{I\,d\mathbf{l}\times\hat{\mathbf{R}}}{4\pi R^{2}}
\qquad
(\hat{\mathbf{R}}=\mathbf{R}/R)
\]

En coordenadas cilíndricas \(({\rho},\phi,z)\) para un filamento sobre el eje \(z\),
\[
d\mathbf{H}= \frac{I\,d\mathbf{l}}{4\pi R^{2}}
             \,\hat{\boldsymbol{a}}_{{\rho}}
             \times\hat{\mathbf{R}}
\]

Biot-Savart expresiones para distribución de carga 

\[
I=\int KdN, \text{ dN perpendicular a la direccion de la corriente.}
\]
\[ Id\mathbf{L}=\mathbf{K}dS=\mathbf{J}dv
\]


% =========================
\subsection*{Contribuciones de corrientes}
% =========================
\begin{align}
\mathbf{H}(\mathbf{r})
  &=\frac{I}{4\pi}\int_{\text{línea}}
      \frac{d\mathbf{l}\times\hat{\mathbf{R}}}{R^{2}}
      &&\text{(corriente lineal \(I\))}\\[6pt]
  &=\frac{1}{4\pi}\int_{\text{superficie}}
      \frac{\mathbf{K}\times\hat{\mathbf{R}}}{R^{2}}\,
      dS
      &&\text{(corriente superficial \(\mathbf{K}\))}\\[6pt]
  &=\frac{1}{4\pi}\int_{\text{volumen}}
      \frac{\mathbf{J}\times\hat{\mathbf{R}}}{R^{2}}\,
      dV
      &&\text{(corriente volumétrica \(\mathbf{J}\))}
\end{align}

\subsection*{Fuerza de Lorentz}
\begin{align}
\mathbf{F}=Q\,\mathbf{u}\times\mathbf{B}
\end{align}

% =========================
\section{Campo \(\mathbf{H}\) de un filamento finito en el eje \(z\)}
% =========================
Sea un conductor recto de longitud \(2L\) centrado en el origen,
transporte de corriente \(I\) hacia \(+z\).
Para un punto \(P({\rho},0,0)\) en el plano \(xy\):

\[
\mathbf{H}(r)=
\frac{I}{4\pi {\rho}}\left(
  \sin\alpha_{2}-\sin\alpha_{1}
\right)\,\hat{\boldsymbol{a}}_{\phi},
\]
donde
\(\alpha_{1}\) del punto a parte superior y \(\alpha_{2}\) del punto a parte inferior del segmento \(P\)‑extremo y el eje \(r\).

\textbf{Casos límite}
\begin{itemize}
  \item {\bf Semi‑infinito} \((L\to\infty\) solo en \(+z)\):
        \(\alpha_{1}=0,\;\alpha_{2}=\pi/2\)
        \[
        \boxed{\mathbf{H}(r)=\dfrac{I}{4\pi {\rho}}\,\hat{\boldsymbol{a}}_{\phi}}
        \]
  \item {\bf Infinito} \((L\to\infty\) en ambas direcciones):
        \(\alpha_{1}=-\pi/2,\;\alpha_{2}=+\pi/2\)
        \[
        \boxed{\mathbf{H}(r)=\dfrac{I}{2\pi {\rho}}\,\hat{\boldsymbol{a}}_{\phi}}
        \]
\end{itemize}

% =========================
\section{Ley de Ampère}
% =========================
\subsection*{Enunciado}
La ley de Ampère establece que la integral de línea de \(\mathbf{H}\)
alrededor de un camino cerrado equivale a la corriente neta \(I_{\mathrm{enc}}\)
encerrada por dicho camino:
\[
\oint_{\mathcal{C}}\mathbf{H}\cdot d\mathbf{L}= I_{\mathrm{enc}}.
\]

\subsection*{Forma inversa}
\[
I_{\mathrm{enc}}=\oint_{\mathcal{C}}\mathbf{H}\cdot d\mathbf{L}.
\]

% =========================
% =========================
\subsection*{Aplicaciones típicas}
% =========================
\begin{itemize}
  \item \textbf{Hilo recto infinito}  
        \[
        \mathbf{H} = \frac{I}{2\pi {\rho}}\,\hat{\boldsymbol{a}}_{\phi} \quad [\text{A/m}]
        \]
        Usando contorno circular alrededor del hilo.

  \item \textbf{Solenoide largo}  
        \[
        \mathbf{H} = nI\,\hat{\boldsymbol{a}}_{z} \quad [\text{A/m}]
        \]
        Donde \(n\) es el número de vueltas por unidad de longitud.

  \item \textbf{Toroide}  
        \[
        \mathbf{H} = \frac{NI}{2\pi {\rho}}\,\hat{\boldsymbol{a}}_{\phi} \quad [\text{A/m}]
        \]
        Siendo \(N\) el número total de espiras.

  \item \textbf{Lámina infinita de corriente (densidad superficial \(K\))}  
        \[
        \mathbf{H} = \frac{K}{2}\,\hat{\boldsymbol{a}}_{\parallel} \quad [\text{A/m}]
        \]
        Donde \(K\) es la densidad de corriente superficial en [A/m] y el campo es paralelo a la lámina, orientado según la regla de la mano derecha.

  \item \textbf{Línea de transmisión coaxial infinita}  
        \[
        \mathbf{H}({\rho}) = 
        \begin{cases}
        \dfrac{I\,{\rho}}{2\pi a^{2}}\,\hat{\boldsymbol{a}}_{\phi} & \text{si } {\rho} < a \\
        \dfrac{I}{2\pi {\rho}}\,\hat{\boldsymbol{a}}_{\phi} & \text{si } a < {\rho} < b \\
        0 & \text{si } {\rho} > b
        \end{cases}
        \quad [\text{A/m}]
        \]
        Donde \(a\) y \(b\) son los radios interno y externo, respectivamente, de los conductores.
\end{itemize}

% =========================
\section{Flujo de Campo Magnético}
% =========================
\subsection*{Definición}
El flujo magnético total que atraviesa una superficie \(S\) es:
\[
\Phi_{B} = \int_{S} \mathbf{B} \cdot d\mathbf{S}
\quad [\text{Wb}]
\]
Donde:
\begin{itemize}
  \item \boxed{\mathbf{B} = \mu_0 \mathbf{H})} es la densidad de flujo magnético \([\text{T} = \text{Wb/m}^2]\).
  \item \(d\mathbf{S}\) es el vector diferencial de área orientado perpendicular a la superficie.
\end{itemize}

\subsection*{Propiedad fundamental}
\[
\nabla \cdot \mathbf{B} = 0 \quad \Rightarrow \quad
\oint_{S} \mathbf{B} \cdot d\mathbf{S} = 0
\]
El flujo total a través de una superficie cerrada es cero. No existen monopolos magnéticos.


% =========================
\section{Potencial Magnetico \(\mathbf{A}\) y $V_m$ }
% =========================
\subsection{Potencial magnético escalar $V_{m}$:}

\[\mathbf{H}= -\nabla V_m,\qquad si \qquad\mathbf{J} = 0\]

\subsection{Potencial magnético vector \(\mathbf{A}\):}

\[\]
\[\mathbf{A}= \int_{L}\frac{\mu Id\mathbf{l}}{4\pi R}
\]

\[
\mathbf{B}=\mu_0\mathbf{H}=\nabla\times\mathbf{A},
\qquad
\mathbf{A}(\mathbf{r})=\frac{\mu_0}{4\pi}\int
\frac{\mathbf{J}\,dV}{R}.
\]
Para un hilo infinito:
\[
\mathbf{A}= \frac{\mu_0 I}{2\pi}\ln\!\left(\frac{r}{r_0}\right)\,
\hat{\boldsymbol{a}}_{\phi}.
\]

\newpage

\section{Divergencia, Gradiente y Laplaciano}

\subsection{Divergencia}

\textbf{Cartesianas:}
\begin{equation}
    \nabla \cdot \mathbf{D} = \frac{\partial D_x}{\partial x} + \frac{\partial D_y}{\partial y} + \frac{\partial D_z}{\partial z}
\end{equation}

\textbf{Cilíndricas:}
\begin{equation}
    \nabla \cdot \mathbf{D} = \frac{1}{\rho} \frac{\partial}{\partial \rho} (\rho D_\rho) + \frac{1}{\rho} \frac{\partial D_\phi}{\partial \phi} + \frac{\partial D_z}{\partial z}
\end{equation}

\textbf{Esféricas:}
\begin{equation}
    \nabla \cdot \mathbf{D} = \frac{1}{r^2} \frac{\partial}{\partial r} (r^2 D_r) + \frac{1}{r \sin \theta} \frac{\partial}{\partial \theta} (\sin \theta D_\theta) + \frac{1}{r \sin \theta} \frac{\partial D_\phi}{\partial \phi}
\end{equation}

\subsection{Gradiente}

\textbf{Cartesianas (rectangulares):}
\begin{equation}
    \nabla V = \frac{\partial V}{\partial x} \, \mathbf{a}_x + \frac{\partial V}{\partial y} \, \mathbf{a}_y + \frac{\partial V}{\partial z} \, \mathbf{a}_z
\end{equation}

\textbf{Cilíndricas:}
\begin{equation}
    \nabla V = \frac{\partial V}{\partial \rho} \, \mathbf{a}_\rho + \frac{1}{\rho} \frac{\partial V}{\partial \phi} \, \mathbf{a}_\phi + \frac{\partial V}{\partial z} \, \mathbf{a}_z
\end{equation}

\textbf{Esféricas:}
\begin{equation}
    \nabla V = \frac{\partial V}{\partial r} \, \mathbf{a}_r + \frac{1}{r} \frac{\partial V}{\partial \theta} \, \mathbf{a}_\theta + \frac{1}{r \sin \theta} \frac{\partial V}{\partial \phi} \, \mathbf{a}_\phi
\end{equation}


\subsection{Laplaciano}

\textbf{Cartesianas:}
\begin{equation}
    \nabla^2 V = \frac{\partial^2 V}{\partial x^2} + \frac{\partial^2 V}{\partial y^2} + \frac{\partial^2 V}{\partial z^2}
\end{equation}

\textbf{Cilíndricas:}
\begin{equation}
    \nabla^2 V = \frac{1}{\rho} \frac{\partial}{\partial \rho} \left( \rho \frac{\partial V}{\partial \rho} \right) + \frac{1}{\rho^2} \frac{\partial^2 V}{\partial \phi^2} + \frac{\partial^2 V}{\partial z^2}
\end{equation}

\textbf{Esféricas:}
\begin{equation}
    \nabla^2 V = \frac{1}{r^2} \frac{\partial}{\partial r} \left( r^2 \frac{\partial V}{\partial r} \right) + \frac{1}{r^2 \sin \theta} \frac{\partial}{\partial \theta} \left( \sin \theta \frac{\partial V}{\partial \theta} \right) + \frac{1}{r^2 \sin^2 \theta} \frac{\partial^2 V}{\partial \phi^2}
\end{equation}

\subsection{Rotacional de un vector}

\textbf{Cartesianas:}
\[
\nabla \times \vec{A} =
\left( \frac{\partial A_z}{\partial y} - \frac{\partial A_y}{\partial z} \right) \hat{x}
+ \left( \frac{\partial A_x}{\partial z} - \frac{\partial A_z}{\partial x} \right) \hat{y}
+ \left( \frac{\partial A_y}{\partial x} - \frac{\partial A_x}{\partial y} \right) \hat{z}
\]


\textbf{Cilíndricas:}

\[
\nabla \times \vec{A} =
\left( \frac{1}{\rho} \frac{\partial A_z}{\partial \phi} - \frac{\partial A_\phi}{\partial z} \right) \hat{\rho}
+ \left( \frac{\partial A_\rho}{\partial z} - \frac{\partial A_z}{\partial \rho} \right) \hat{\phi}
+ \frac{1}{\rho} \left( \frac{\partial (\rho A_\phi)}{\partial \rho} - \frac{\partial A_\rho}{\partial \phi} \right) \hat{z}
\]



\textbf{Esféricas:}

\[
\nabla \times \vec{A} =
\frac{1}{r \sin\theta}
\left[
\frac{\partial}{\partial \theta} (A_\phi \sin\theta) - \frac{\partial A_\theta}{\partial \phi}
\right] \hat{r}
+ \frac{1}{r}
\left[
\frac{1}{\sin\theta} \frac{\partial A_r}{\partial \phi} - \frac{\partial}{\partial r}(r A_\phi)
\right] \hat{\theta}
+ \frac{1}{r}
\left[
\frac{\partial}{\partial r}(r A_\theta) - \frac{\partial A_r}{\partial \theta}
\right] \hat{\phi}
\]


\section{Tabla de Productos cruz entre vectores}
\subsection{Rectangulares}
\begin{table}[H]
\centering
\begin{tabular}{c|ccc}
$\times$ & $\hat{x}$ & $\hat{y}$ & $\hat{z}$ \\
\hline
$\hat{x}$ & $0$ & $\hat{z}$ & $-\hat{y}$ \\
$\hat{y}$ & $-\hat{z}$ & $0$ & $\hat{x}$ \\
$\hat{z}$ & $\hat{y}$ & $-\hat{x}$ & $0$
\end{tabular}
\caption{Producto cruzado en coordenadas rectangulares}
\end{table}
\subsection{Cilíndricas}

\begin{table}[H]
\centering
\begin{tabular}{c|ccc}
$\times$ & $\hat{\rho}$ & $\hat{\phi}$ & $\hat{z}$ \\
\hline
$\hat{\rho}$ & $0$ & $\hat{z}$ & $-\hat{\phi}$ \\
$\hat{\phi}$ & $-\hat{z}$ & $0$ & $\hat{\rho}$ \\
$\hat{z}$ & $\hat{\phi}$ & $-\hat{\rho}$ & $0$
\end{tabular}
\caption{Producto cruzado en coordenadas cilíndricas}
\end{table}


\subsection{Esféricas}

\begin{table}[H]
\centering
\begin{tabular}{c|ccc}
$\times$ & $\hat{r}$ & $\hat{\theta}$ & $\hat{\phi}$ \\
\hline
$\hat{r}$ & $0$ & $\hat{\phi}$ & $-\hat{\theta}$ \\
$\hat{\theta}$ & $-\hat{\phi}$ & $0$ & $\hat{r}$ \\
$\hat{\phi}$ & $\hat{\theta}$ & $-\hat{r}$ & $0$
\end{tabular}
\caption{Producto cruzado en coordenadas esféricas}
\end{table}

\subsection{Combinaciones}
\begin{table}[H]
\centering
\renewcommand{\arraystretch}{1.4}
\begin{tabular}{c|ccc}
$\times$ & $\hat{x}$ & $\hat{y}$ & $\hat{z}$ \\
\hline
$\hat{\rho}$ & $-\sin\phi\, \hat{z}$ & $\cos\phi\, \hat{z}$ & $\cos\phi\, \hat{y} - \sin\phi\, \hat{x}$ \\
$\hat{\phi}$ & $\cos\phi\, \hat{z}$ & $\sin\phi\, \hat{z}$ & $\sin\phi\, \hat{x} + \cos\phi\, \hat{y}$ \\
$\hat{z}$ (cil) & $\hat{y}$ & $-\hat{x}$ & $0$ \\
\hline
$\hat{r}$ & $ \sin\theta\sin\phi\, \hat{z} - \cos\theta\, \hat{y} $ & $ \cos\theta\, \hat{x} - \sin\theta\cos\phi\, \hat{z} $ & $ \sin\theta \cos\phi\, \hat{y} - \sin\theta \sin\phi\, \hat{x} $ \\
$\hat{\theta}$ & $ \cos\theta\sin\phi\, \hat{z} + \sin\theta\, \hat{y} $ & $ -\sin\theta\, \hat{x} - \cos\theta\cos\phi\, \hat{z} $ & $ \cos\theta \cos\phi\, \hat{y} - \cos\theta \sin\phi\, \hat{x} $ \\
$\hat{\phi}$ (esf) & $ \hat{y} $ & $ -\hat{x} $ & $0$
\end{tabular}
\caption{Productos cruzados mixtos entre coordenadas rectangulares, cilíndricas y esféricas}
\end{table}


\newpage

\begin{multicols}{2}

\section{Cambios de variable para coordenadas}

\section{Diferenciales en Coordenadas}

\subsection{Rectangulares}
\begin{align}
    d\mathbf{l} &= \hat{a}_x \, dx + \hat{a}_y \, dy + \hat{a}_z \, dz \\[6pt]
    d\mathbf{S} &= \begin{cases}
        \hat{a}_z \, dx \, dy \\
        \hat{a}_y \, dz \, dx \\
        \hat{a}_x \, dy \, dz
    \end{cases} \\[6pt]
    dV &= dx \, dy \, dz
\end{align}

\subsection{Cilíndricas}
\begin{align}
    d\mathbf{l} &= \hat{a}_\rho \, d\rho + \hat{a}_\theta \, \rho \, d\theta + \hat{a}_z \, dz \\[6pt]
    d\mathbf{S} &= \begin{cases}
        \hat{a}_\rho \, \rho \, d\theta \, dz \\
        \hat{a}_\theta \, d\rho \, dz \\
        \hat{a}_z \, \rho \, d\rho \, d\theta
    \end{cases} \\[6pt]
    dV &= \rho \, d\rho \, d\theta \, dz
\end{align}

\subsection{Esféricas}
\begin{align}
    d\mathbf{l} &= \hat{a}_r \, dr + \hat{a}_\theta \, r \, d\theta + \hat{a}_\phi \, r \sin \theta \, d\phi \\[6pt]
    d\mathbf{S} &= \begin{cases}
        \hat{a}_r \, r^2 \sin\theta \, d\theta \, d\phi \\
        \hat{a}_\theta \, r \, dr \, d\phi \\
        \hat{a}_\phi \, r \, dr \, d\theta
    \end{cases} \\[6pt]
    dV &= r^2 \sin\theta \, dr \, d\theta \, d\phi
\end{align}
\columnbreak

\section{1. Rectangulares a Cilíndricas}
\begin{equation}
\rho = \sqrt{x^2 + y^2} \qquad 
\phi = \tan^{-1}\left( \frac{y}{x} \right) \qquad 
z = z
\end{equation}


\begin{equation}
    \hat{a}_\rho = \hat{a}_x \cos \phi + \hat{a}_y \sin \phi
\end{equation}
\begin{equation}
    \hat{a}_\phi = -\hat{a}_x \sin \phi + \hat{a}_y \cos \phi
\end{equation}
\begin{equation}
    \hat{a}_z = \hat{a}_z
\end{equation}

\section{2. Cilíndricas a Rectangulares}
\begin{equation}
x = \rho \cos \phi \qquad 
y = \rho \sin \phi \qquad 
z = z
\end{equation}


\begin{equation}
    \hat{a}_x = \hat{a}_\rho \cos \phi + \hat{a}_\phi \sin \phi
\end{equation}
\begin{equation}
    \hat{a}_y = -\hat{a}_\rho \sin \phi + \hat{a}_\phi \cos \phi
\end{equation}
\begin{equation}
    \hat{a}_z = \hat{a}_z
\end{equation}

\section{3. Rectangulares a Esféricas}
\begin{equation}
r = \sqrt{x^2 + y^2 + z^2} \qquad 
\theta = \cos^{-1} \left( \frac{z}{r} \right) \qquad 
\phi = \tan^{-1} \left( \frac{y}{x} \right)
\end{equation}


\begin{equation}
    \hat{a}_r = \hat{a}_x \sin \theta \cos \phi + \hat{a}_y \sin \theta \sin \phi + \hat{a}_z \cos \theta
\end{equation}
\begin{equation}
    \hat{a}_\theta = \hat{a}_x \cos \theta \cos \phi + \hat{a}_y \cos \theta \sin \phi - \hat{a}_z \sin \theta
\end{equation}
\begin{equation}
    \hat{a}_\phi = -\hat{a}_x \sin \phi + \hat{a}_y \cos \phi
\end{equation}

\section{4. Esféricas a Rectangulares}
\begin{equation}
x = r \sin \theta \cos \phi \qquad 
y = r \sin \theta \sin \phi \qquad 
z = r \cos \theta
\end{equation}


\begin{equation}
    \hat{a}_x = \hat{a}_r \sin \theta \cos \phi + \hat{a}_\theta \cos \theta \cos \phi - \hat{a}_\phi \sin \phi
\end{equation}
\begin{equation}
    \hat{a}_y = \hat{a}_r \sin \theta \sin \phi + \hat{a}_\theta \cos \theta \sin \phi + \hat{a}_\phi \cos \phi
\end{equation}
\begin{equation}
    \hat{a}_z = \hat{a}_r \cos \theta - \hat{a}_\theta \sin \theta
\end{equation}

\end{multicols}


\newpage

\section{Figuras Geométricas: Perímetro, Área y Volumen}

\subsection{Círculo}
\begin{equation*}
\begin{array}{ll}
\text{Perímetro (circunferencia):} & P = 2\pi r \\[5pt]
\text{Área:} & A = \pi r^2
\end{array}
\end{equation*}

\subsection{Cilindro}
\begin{equation*}
\begin{array}{ll}
\text{Área lateral:} & A_L = 2\pi r h \\[5pt]
\text{Área total:} & A_T = 2\pi r h + 2\pi r^2 \\[5pt]
\text{Volumen:} & V = \pi r^2 h
\end{array}
\end{equation*}

\subsection{Esfera}
\begin{equation*}
\begin{array}{ll}
\text{Área superficial:} & A = 4\pi r^2 \\[5pt]
\text{Volumen:} & V = \frac{4}{3} \pi r^3
\end{array}
\end{equation*}

\subsection{Cono}
\begin{equation*}
\begin{array}{ll}
\text{Área lateral:} & A_L = \pi r l \\[5pt]
\text{Área total:} & A_T = \pi r l + \pi r^2 \\[5pt]
\text{Volumen:} & V = \frac{1}{3} \pi r^2 h
\end{array}
\end{equation*}

\subsection{Cubo}
\begin{equation*}
\begin{array}{ll}
\text{Área total:} & A_T = 6a^2 \\[5pt]
\text{Volumen:} & V = a^3    
\end{array}
\end{equation*}

\subsection{Prisma Rectangular (Paralelepípedo)}
\begin{equation*}
\begin{array}{ll}
\text{Área total:} & A_T = 2(ab + bc + ac) \\[5pt]
\text{Volumen:} & V = abc
\end{array}
\end{equation*}

\subsection{Polígono Regular de \(n\) lados}
\begin{equation*}
\begin{array}{ll}
    \text{Perímetro} \quad P &= n \cdot l \\[5pt]
    \text{Apotema} \quad a &= \frac{l}{2 \tan\left( \frac{\pi}{n} \right)} \\[5pt]
    \text{Área} \quad A &= \frac{n \cdot l \cdot a}{2} = \frac{n l^2}{4 \tan\left( \frac{\pi}{n} \right)}
\end{array}
\end{equation*}

\newpage



\section*{Integrales Comunes}

\begin{align}
    \int x^n \, dx &= \frac{x^{n+1}}{n+1} + C, \quad n \neq -1 \\[10pt]
    \int \frac{dx}{x} &= \ln|x| + C \\[10pt]
    \int e^{ax} \, dx &= \frac{e^{ax}}{a} + C \\[10pt]
    \int a^{x} \, dx &= \frac{a^{x}}{\ln(a)} + C, \quad a > 0, a \neq 1 \\[10pt]
    \int \ln(x) \, dx &= x \ln(x) - x + C
\end{align}

\section*{Integrales Trigonométricas}


\begin{align}
    \int \sin(ax) \, dx &= -\frac{1}{a} \cos(ax) + C \\[10pt]
    \int \cos(ax) \, dx &= \frac{1}{a} \sin(ax) + C \\[10pt]
    \int \tan(ax) \, dx &= \frac{1}{a} \ln|\sec(ax)| + C \\[10pt]
    \int \cot(ax) \, dx &= \frac{1}{a} \ln|\sin(ax)| + C \\[10pt]
    \int \sec(ax) \, dx &= \frac{1}{a} \ln|\sec(ax) + \tan(ax)| + C \\[10pt]
    \int \frac{1}{\sin(x)} \, dx &= \int \csc(x) \, dx = \ln \left| \tan\left( \frac{x}{2} \right) \right| + C \\[10pt]
    \int \sin^2(ax) \, dx &= \frac{x}{2} - \frac{\sin(2ax)}{4a} + C \\[10pt]
    \int \cos^2(ax) \, dx &= \frac{x}{2} + \frac{\sin(2ax)}{4a} + C
\end{align}


\section*{Integrales que Involucran Funciones Inversas}

\begin{align}
    \int \frac{dx}{\sqrt{a^2 - x^2}} &= \sin^{-1}\left(\frac{x}{a}\right) + C \\[10pt]
    \int \frac{dx}{\sqrt{x^2 - a^2}} &= \cosh^{-1}\left(\frac{x}{a}\right) + C \\[10pt]
    \int \frac{dx}{a^2 + x^2} &= \frac{1}{a} \tan^{-1}\left(\frac{x}{a}\right) + C \\[10pt]
    \int \frac{dx}{x \sqrt{x^2 - a^2}} &= \frac{1}{a} \sec^{-1}\left|\frac{x}{a}\right| + C
\end{align}

\section*{Integrales que Requieren Sustituciones Específicas}

\begin{align}
    \int \sqrt{a^2 - x^2} \, dx &= \frac{x}{2} \sqrt{a^2 - x^2} + \frac{a^2}{2} \sin^{-1}\left(\frac{x}{a}\right) + C \\[10pt]
    \int \sqrt{x^2 + a^2} \, dx &= \frac{x}{2} \sqrt{x^2 + a^2} + \frac{a^2}{2} \ln\left| x + \sqrt{x^2 + a^2} \right| + C \\[10pt]
    \int \sqrt{x^2 - a^2} \, dx &= \frac{x}{2} \sqrt{x^2 - a^2} - \frac{a^2}{2} \ln\left| x + \sqrt{x^2 - a^2} \right| + C 
\end{align}

\begin{equation}
\int \frac{1}{(x + b)\sqrt{x + c}} \, dx = 
\frac{2}{\sqrt{c - b}} \ln \left| 
\frac{ \sqrt{x + c} + \sqrt{c - b} }{ \sqrt{x + c} - \sqrt{c - b} } 
\right| + C
\end{equation}


\begin{equation}
    \int \frac{dx}{\sqrt{x^2 + a^2}} = \ln\left| x + \sqrt{x^2 + a^2} \right| + C
\end{equation}

\begin{equation}
    \int \frac{dx}{\sqrt{x^2 - a^2}} = \ln\left| x + \sqrt{x^2 - a^2} \right| + C
\end{equation}

\section*{Integrales con Productos de Funciones Trigonométricas}

\begin{align}
    \int x \sin(ax) \, dx &= -\frac{x}{a} \cos(ax) + \frac{1}{a^2} \sin(ax) + C \\[10pt]
    \int x \cos(ax) \, dx &= \frac{x}{a} \sin(ax) + \frac{1}{a^2} \cos(ax) + C \\[10pt]
    \int \sin(ax) \sin(bx) \, dx &= \frac{1}{2} \int [\cos((a-b)x) - \cos((a+b)x)] \, dx \\[10pt]
    &= \frac{1}{2} \left( \frac{\sin((a-b)x)}{a-b} - \frac{\sin((a+b)x)}{a+b} \right) + C, \quad a \neq b \\[10pt]
    \int \cos(ax) \cos(bx) \, dx &= \frac{1}{2} \int [\cos((a-b)x) + \cos((a+b)x)] \, dx \\[10pt]
    &= \frac{1}{2} \left( \frac{\sin((a-b)x)}{a-b} + \frac{\sin((a+b)x)}{a+b} \right) + C, \quad a \neq b \\[10pt]
    \int \sin(ax) \cos(bx) \, dx &= \frac{1}{2} \int [\sin((a+b)x) + \sin((a-b)x)] \, dx \\[10pt]
    &= \frac{1}{2} \left( -\frac{\cos((a+b)x)}{a+b} + \frac{\cos((a-b)x)}{a-b} \right) + C, \quad a \neq b
\end{align}

\section*{Integral con Expresión Radical}

\begin{align}
    \int \frac{x}{\sqrt{x^2 + a^2}} \, dx &= \sqrt{x^2 + a^2} + C \\[10pt]
    \int \frac{dx}{(x^2 + a^2)^{3/2}} &= \frac{x}{a^2 \sqrt{x^2 + a^2}} + C \\[10pt]
    \int \frac{x \, dx}{(x^2 + a^2)^{3/2}} &= -\frac{1}{\sqrt{x^2 + a^2}} + C
\end{align}


\section*{Integrales de Funciones Exponenciales}

\begin{align}
    \int_0^{\infty} e^{-ax} \, dx &= \frac{1}{a}, \quad a > 0 \\[10pt]
    \int_0^{\infty} x e^{-ax} \, dx &= \frac{1}{a^2}, \quad a > 0 \\[10pt]
    \int_0^{\infty} x^2 e^{-ax} \, dx &= \frac{2}{a^3}, \quad a > 0
\end{align}

\section*{Integrales Especiales}

\begin{align}
    \int_0^{\infty} \frac{\sin x}{x} \, dx &= \frac{\pi}{2} \\[10pt]
    \int_0^{\infty} \frac{\cos x}{x} \, dx &= 0 \\[10pt]
    \int_0^{\pi/2} \ln (\cos x) \, dx &= -\frac{\pi}{2} \ln 2 \\[10pt]
    \int_0^{\pi/2} \ln (\sin x) \, dx &= -\frac{\pi}{2} \ln 2 \\[10pt]
    \int_0^{\infty} e^{-x^2} \, dx &= \frac{\sqrt{\pi}}{2} \\[10pt]
    \int_{-\infty}^{\infty} e^{-x^2} \, dx &= \sqrt{\pi} \\[10pt]
    \int_0^{\infty} \frac{x}{(x^2 + a^2)^{3/2}} \, dx &= \frac{1}{a} \\[10pt]
    \int_0^{\infty} \frac{dx}{(x^2 + a^2)^{3/2}} \, dx &= \frac{1}{a^2}
\end{align}




\end{document}
